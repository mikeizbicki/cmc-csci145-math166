\documentclass[10pt]{article}

\usepackage[margin=1in]{geometry}
\usepackage{amsmath}
\usepackage{amssymb}
\usepackage{amsthm}
\usepackage{mathtools}
\usepackage[shortlabels]{enumitem}

\usepackage{hyperref}
\hypersetup{
  colorlinks   = true, %Colours links instead of ugly boxes
  urlcolor     = black, %Colour for external hyperlinks
  linkcolor    = blue, %Colour of internal links
  citecolor    = blue  %Colour of citations
}

%%%%%%%%%%%%%%%%%%%%%%%%%%%%%%%%%%%%%%%%%%%%%%%%%%%%%%%%%%%%%%%%%%%%%%%%%%%%%%%%

\theoremstyle{definition}
\newtheorem{problem}{Problem}
\newcommand{\E}{\mathbb E}
\newcommand{\R}{\mathbb R}
\DeclareMathOperator{\Var}{Var}
\DeclareMathOperator*{\argmin}{arg\,min}
\DeclareMathOperator*{\argmax}{arg\,max}

\newcommand{\trans}[1]{{#1}^{T}}
\newcommand{\loss}{\ell}
\newcommand{\w}{\mathbf w}
\newcommand{\mle}[1]{\hat{#1}_{\textit{mle}}}
\newcommand{\map}[1]{\hat{#1}_{\textit{map}}}
\newcommand{\normal}{\mathcal{N}}
\newcommand{\x}{\mathbf x}
\newcommand{\y}{\mathbf y}
\newcommand{\ltwo}[1]{\lVert {#1} \rVert}

%Problem ideas:
%second derivative of the log-loss
%calculate the minimum of the OLS objective
%definition of strictly convex equivalent to second derivative positive semidefinite, first derivative condition
%jensen's inequality: arithemetic mean never less than geometric mean

%%%%%%%%%%%%%%%%%%%%%%%%%%%%%%%%%%%%%%%%%%%%%%%%%%%%%%%%%%%%%%%%%%%%%%%%%%%%%%%%

\begin{document}


\begin{center}
    {
\Large
Notes: maximum a posteriori estimation
}

    \vspace{0.1in}
CSCI145/MATH166, Mike Izbicki

    \vspace{0.1in}
\end{center}

\begin{problem}
    The editor for a school newspaper wants to predict the number of readers $x$ who will ``like'' a given article on facebook.
    The poisson distribution is often used to model the number of times an event occurs,
    and so the editor chooses to model $x$ as a poisson distributed random variable.
    That is, 
    \begin{equation}
        p(x|\lambda) = \frac{\lambda^x \mathrm{e}^{-\lambda}}{x!}
        .
    \end{equation}
    %In problem \ref{prob:poisson}, you calculated the maximum likelihood estimator for $\lambda$.
    The editor is not satisfied with maximum likelihood estimation and prefers to use MAP estimation with a gamma prior on $\lambda$.
    The gamma distribution has density
    \begin{equation}
        p(\lambda|\alpha,\beta) = \frac{\beta^\alpha \lambda^{\alpha-1}\exp(-\beta\lambda)}{\Gamma(\alpha)}
        ,
    \end{equation}
    where $\beta\ge0$ and $\alpha\ge1$ are hyperparameters, and 
    \begin{equation}
        \Gamma(\alpha) = \int_0^\infty x^{\alpha-1}\exp(-\alpha) dx
    \end{equation}
    is the gamma function.
    Calculate the MAP estimate of $\lambda$.
\end{problem}

\end{document}

