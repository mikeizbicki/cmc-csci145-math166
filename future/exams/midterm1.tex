\documentclass[10pt]{exam}

\usepackage[margin=1in]{geometry}
\usepackage{amsmath}
\usepackage{amssymb}
\usepackage{amsthm}
\usepackage{mathtools}
\usepackage{bm}
\usepackage[normalem]{ulem}

\usepackage{color}
\usepackage{colortbl}
\definecolor{deepblue}{rgb}{0,0,0.5}
\definecolor{deepred}{rgb}{0.6,0,0}
\definecolor{deepgreen}{rgb}{0,0.5,0}
\definecolor{gray}{rgb}{0.7,0.7,0.7}

\usepackage{hyperref}
\hypersetup{
  colorlinks   = true, %Colours links instead of ugly boxes
  urlcolor     = black, %Colour for external hyperlinks
  linkcolor    = blue, %Colour of internal links
  citecolor    = blue  %Colour of citations
}

%%%%%%%%%%%%%%%%%%%%%%%%%%%%%%%%%%%%%%%%%%%%%%%%%%%%%%%%%%%%%%%%%%%%%%%%%%%%%%%%

\newcommand*{\hl}[1]{\colorbox{yellow}{#1}}

\newcommand*{\answerLong}[2]{
    \ifprintanswers{\hl{#1}}
\else{#2}
\fi
}

\newcommand*{\answer}[1]{\answerLong{#1}{~}}

\newcommand*{\TrueFalse}[1]{%
\ifprintanswers
    \ifthenelse{\equal{#1}{T}}{%
        %\hl{\textbf{TRUE}}\hspace*{14pt}False
        \hl{\texttt{True}}\hspace*{20pt}\texttt{False}\hspace*{20pt}\texttt{Open}
    }{
        \ifthenelse{\equal{#1}{F}}{
        %True\hspace*{14pt}\hl{\textbf{FALSE}}
        \texttt{True}\hspace*{20pt}\hl{\texttt{False}}\hspace*{20pt}\texttt{Open}
        }
        {
            \texttt{True}\hspace*{20pt}{\texttt{False}}\hspace*{20pt}\hl{\texttt{Open}}
        }
    }
\else
    \texttt{True}\hspace*{20pt}\texttt{False}\hspace*{20pt}\texttt{Open}
\fi
} 
%% The following code is based on an answer by Gonzalo Medina
%% https://tex.stackexchange.com/a/13106/39194
\newlength\TFlengthA
\newlength\TFlengthB
\settowidth\TFlengthA{\hspace*{1.8in}}
\newcommand\TFQuestion[2]{%
    \setlength\TFlengthB{\linewidth}
    \addtolength\TFlengthB{-\TFlengthA}
    \noindent
    \parbox[t]{\TFlengthA}{\TrueFalse{#1}}\parbox[t]{\TFlengthB}{#2}
    \vspace{0.25in}
}

%%%%%%%%%%%%%%%%%%%%%%%%%%%%%%%%%%%%%%%%%%%%%%%%%%%%%%%%%%%%%%%%%%%%%%%%%%%%%%%%

\theoremstyle{definition}
\newtheorem{problem}{Problem}
\newtheorem{theorem}{Theorem}
\newtheorem{defn}{Definition}
\newtheorem{refr}{References}
\newcommand{\E}{\mathbb E}
\newcommand{\R}{\mathbb R}
\DeclareMathOperator{\nnz}{nnz}
\DeclareMathOperator{\determinant}{det}
\DeclareMathOperator{\Var}{Var}
\DeclareMathOperator{\rank}{rank}
\DeclareMathOperator*{\argmin}{arg\,min}
\DeclareMathOperator*{\argmax}{arg\,max}

\newcommand{\I}{\mathbf I}
\newcommand{\Q}{\mathbf Q}
\newcommand{\p}{\mathbf P}
\newcommand{\pb}{\bar {\p}}
\newcommand{\pbb}{\bar {\pb}}
\newcommand{\pr}{\bm \pi}

\newcommand{\trans}[1]{{#1}^{T}}
\newcommand{\loss}{\ell}
\newcommand{\w}{\mathbf w}
\newcommand{\x}{\mathbf x}
\newcommand{\y}{\mathbf y}
\newcommand{\lone}[1]{{\lVert {#1} \rVert}_1}
\newcommand{\ltwo}[1]{{\lVert {#1} \rVert}_2}
\newcommand{\lp}[1]{{\lVert {#1} \rVert}_p}
\newcommand{\linf}[1]{{\lVert {#1} \rVert}_\infty}
\newcommand{\lF}[1]{{\lVert {#1} \rVert}_F}

\newcommand{\ignore}[1]{}

%%%%%%%%%%%%%%%%%%%%%%%%%%%%%%%%%%%%%%%%%%%%%%%%%%%%%%%%%%%%%%%%%%%%%%%%%%%%%%%%

\printanswers

\begin{document}


\begin{center}
{
\Huge
    Midterm 1
}
\end{center}

\vspace{0.5in}
\noindent
\textbf{Printed Name:}

\noindent
\rule{\textwidth}{0.1pt}
\vspace{0.25in}

%Format:
%
    %The midterm will be worth 15 points.
    %There will be 10 True / False / Open questions. Each correct answer will be worth 1 point, each incorrect answer will result in a -1 penalty, and each blank answer will result in 0 points.
    %There will be 5 free response questions. These questions will follow the format of the questions in the notes packets.
%
    \noindent
\textbf{Due date:}
\begin{enumerate}
    \item
        The exam is due Monday 26 Sep at 8AM.

    \item
        You may submit it either on sakai electronically or by putting a physical copy under my door.
\end{enumerate}

\vspace{0.15in}
\noindent
\textbf{Rules:}
\begin{enumerate}
    \item
        The exam is untimed.
        You do not have to complete the exam in a single sitting.
        You may pause and restart whenever you'd like.

    \item
    You may use any non-human resources that you like, including notes, books, internet articles, and computers.

    \item
    You are not allowed to discuss the exam in any way with any human until after the due date. This includes:
        \begin{enumerate}
            \item obviously bad behavior like copying answers, 
            \item more banal behavior such as:
                \begin{enumerate}
                    \item telling your friend "Problem 6 was really hard" or 
                    \item asking your friend "Have you completed the exam yet?"
                \end{enumerate}
        \end{enumerate}
        
        Even after you finish the exam, you may not discuss it.

    \item
    If you do have questions about the exam, you should email me the questions rather than posting to github.

\end{enumerate}

\vspace{0.15in}
\noindent
\textbf{Grading:}
\begin{enumerate}
    \item For the True/False/Open questions: Each correct answer will be awarded +1 point, each incorrect answer will result in a -1 point penalty, and each blank answer will result in 0 points.

    \item All other problems are worth 1 point, with no penalty for incorrect answers.

    \item There are 16 points possible on the exam.
        Your final grade entered into sakai will be
        \begin{equation*}
            \min\{15, \text{the number of points earned}\}.
        \end{equation*}

    \item If you find a substantive error on the exam, then I will award you +1 bonus point.
\end{enumerate}

\vspace{0.15in}
\noindent
\textbf{Good luck :)}

\newpage

\begin{problem}
    For each statement below,
    circle \texttt{True} if the statement is known to be true,
    \texttt{False} if the statement is known to be false,
    and \texttt{Open} if the statement is not known to be either true or false.
    Ensure that you pay careful attention to the formal definitions of asymptotic notation in your responses.

\begin{enumerate}
%\item\TFQuestion{T}{Let $f(n) = 1/(1+n)$. Then $f = \Omega(n^{-2})$.}
    %\item\TFQuestion{T}{There exist $n\times n$ matrices $A$ and $B$ where $\nnz(AB) = 1$ and $\nnz(BA) = n^2$.}
\item\TFQuestion{O}{Let $A$ and $B$ be dense $n\times n$ matrices. Then the fastest possible algorithm for computing the matrix product $AB$ has runtime $\Omega(n^2\log^2 n)$.}
    \item\TFQuestion{T}{Let $A$ be a dense $n\times n$ matrix. Then the fastest possible algorithm for computing the matrix product $A^3$ has runtime $O(n^{3})$.}
%\item\TFQuestion{O}{Let $A$ and $B$ be $n\times n$ matrices. The fastest algorithm for computing the matrix product $AB$ has runtime $\Omega(n^{2.1})$.}
%\item\TFQuestion{T}{Let $A$ and $B$ be $n\times n$ matrices. The fastest algorithm for computing the matrix product $AB$ has runtime $\Omega(n^{2})$.}
%\item\TFQuestion{F}{Let $A$ and $B$ be $n\times n$ matrices. The fastest algorithm for computing the matrix product $AB$ has runtime $\Omega(n^{2.5})$.}
%\item\TFQuestion{T}{Let $A$ be an $n \times n$ matrix and $k$ a natural number. Then the matrix exponential $A^k$ can be computed in time $O(n^{2.807}\log k)$.}
%\item\TFQuestion{O}{Let $A$ be an $n \times n$ matrix and $k$ a natural number. Then the matrix exponential $A^k$ can be computed in time $O(n^{2}\log n\log k)$.}
%\item\TFQuestion{O}{The matrix chain ordering problem can be solved in time $\Theta(n)$.}
\item\TFQuestion{T}{Let $\x$ be a dense $n$-dimensional vector.  Then the fastest possible algorithm for computing the expression $\x\trans\x\x$ has runtime $\Theta(n)$.}
%\item\TFQuestion{T}{For all $n\in\mathbb N$, there exist $n \times n$ matrices $A$ and $B$ with $\nnz(A) = O(1)$ and $\nnz(B) = O(1)$ such that the product $AB$ satisfies $\nnz(AB) = O(1)$.}
%\item\TFQuestion{T}{For all $n \times n$ matrices $A$ and $B$ with $\nnz(A) = O(1)$ and $\nnz(B) = O(1)$, the product $AB$ must satisfy $\nnz(AB) = O(1)$.}
%\item\TFQuestion{F}{There exist $n \times n$ matrices $A$ and $B$ with $\nnz(A) = O(1)$ and $\nnz(B) = O(1)$ such that the product $AB$ satisfies $\nnz(AB) = \Omega(n)$.}
%\item\TFQuestion{F}{There exists an $n \times n$ matrix $A$ such that $\nnz(A) = \Omega(n^3)$.}
%\item\TFQuestion{T}{There exists an $n \times n$ matrix $A$ such that $\nnz(A) = O(n^3)$.}
%\item\TFQuestion{T}{Let $A$ be an $n\times n$ matrix that is both primitive and stochastic. Then $A$ has exactly one eigenvalue equal to 1.}
\item{\sout{\TFQuestion{F}{It is always true that $\ltwo{\pb} = \ltwo{\pbb}$.}}

    (Everyone awarded +1 point for this problem, regardless of your answer.  It's true that the top eigenvalues for both matrices must be equal to 1 because they are both stochastic, but the L2 norm technically measures the top singular value rather than the top eigenvalue, and these need not be equal.)
        }

\item\TFQuestion{T}{Assume we are computing the pagerank of a graph with 10 nodes using $\alpha=0.85$.  Then it is possible for the L1 norm of the pagerank vector to be equal to 1.}
%\item\TFQuestion{F}{The $\p$ matrix is primitive.}
\item\TFQuestion{T}{There exists a valid $\p$ matrix such that $\nnz(\p) = n$ and $\nnz(\p\p) = n$.}
\item\TFQuestion{F}{Every stochastic matrix is irreducible.}
\item\TFQuestion{T}{$\nnz(\pbb) = O(n^3)$.}
%\item\TFQuestion{F}{Assume that $\x$ is a stochastic vector satisfying $\x \ne \pi$. Then $\ltwo{\trans\x\pbb} < \ltwo{\x}$.}
\item\TFQuestion{T}{It is possible for the $\pb$ matrix to be primitive.}
\item\TFQuestion{F}{It is possible for the $\pbb$ matrix to have an eigenvalue $\lambda$ that is greater than $\alpha$ but less than 1; that is, an eigenvalue satisfying $1 > \lambda > \alpha$.}
%\item\TFQuestion{T}{The $\pbb$ matrix is stochastic.}
%\item\TFQuestion{T}{For all graphs, $\nnz(\pbb) = 0$.}
%\item\TFQuestion{F}{For all graphs, the node with the largest in degree must have the highest pagerank.}
%\item\TFQuestion{F}{According to the \emph{Deeper Inside Pagerank} paper, the $\p$ matrix constructed from the web graph satisfies $\nnz(\p) = O(1)$.}
%\item\TFQuestion{F}{According to the \emph{Deeper Inside Pagerank} paper, Google uses a value of $\alpha=0.8$ when computing pagerank.}
%\item\TFQuestion{T}{The L1 norm of the pagerank vector is always 1.  That is, $\lone{\pr} = 1$.}
%\item\TFQuestion{F}{The exponentially accelerated power method is faster than the standard power method for computing the pagerank of sparse graphs with many nodes.}
%\item\TFQuestion{T}{If the power method is taking too long to converge, then increasing the $\alpha$ hyperparameter will make the algorithm run faster.}
%\item\TFQuestion{F}{The number of iterations of the standard pagerank algorithm depends on the choice of personalization vector $\mathbf v$.}
%\item\TFQuestion{T}{Decreasing the $\alpha$ hyperparameter increases the effect of the personalization vector $\mathbf v$ on the computed pagerank.}
%\item\TFQuestion{T}{Decreasing the $\alpha$ hyperparameter will decrease the pagerank of the node with the highest pagerank.}
%\item\TFQuestion{T}{Decreasing the $\alpha$ hyperparameter will increase the pagerank of the node with the lowest pagerank.}
%\item\TFQuestion{F}{Increasing $\alpha$ decreases the subdominant eigenvalue of the $\pbb$ matrix.}
%\item\TFQuestion{F}{Increasing $\alpha$ is guaranteed to make power method require more iterations to converge to the same residual $\epsilon$.}
%\item\TFQuestion{F}{As the number of nodes $n$ in the graph increases (i.e.\ the dimensions of the $\p$ matrix increase), the number of iterations required by the power method will also increase.}
%\item\TFQuestion{F}{The runtime of a single iteration of the power method using Equation (5.1) increases as $\alpha$ increases.}
%\item\TFQuestion{F}{The runtime of a single iteration of the exponentially accelerated power method using Equation (5.1) increases as $\epsilon$ decreases.}
%\item\TFQuestion{T}{If $n$ is small, $\p$ is dense, and you require an extremely accurate estimate of the pagerank, then the exponentially accelerated power method will likely be faster than the standard power method.}
\end{enumerate}
\end{problem}

\newpage
\begin{problem}
Either prove or give a counterexample to the following claim: For any $n$-dimensional vector $\x$, it is true that $\ltwo{\trans\x\pbb} \le \ltwo{\x}$.
\end{problem}
\begin{solution}
    The statement is FALSE, and I awarded full credit for any valid counterexample.
    For example,
    \begin{equation}
        \trans\x = 
        \begin{pmatrix}
            1 & 1 & 1
        \end{pmatrix}
        \qquad\text{and}\qquad \pbb=
        \begin{pmatrix}
            0.5 & 0.25 & 0.25 \\
            0.5 & 0.25 & 0.25 \\
            0.5 & 0.25 & 0.25 \\
        \end{pmatrix}
    \end{equation}
    results in $\ltwo{\x} = \sqrt{3} \approx 1.732$ and $\ltwo{\trans\x\pbb} \approx 1.837$.
\end{solution}

\begin{solution}
    %\vspace{1in}
    I also awarded full credit if you claimed that the answer was TRUE somehow due to $\pbb$ having a maximum eigenvalue of 1.
    That is, if you stated something like:
    %with the following justification:
    %$\pbb$ has a maximum eigenvalue of 1,
    %so
    \begin{align}
        \ltwo{\trans\x\pbb}
        &\le
        \ltwo{\x}\ltwo{\pbb} \\
        &=
        \ltwo{\x}
    \end{align}
    where the first step is true due to the properties of norms and the second step is true because $\ltwo{\pbb} = 1$.

    Unfortunately, it is not true that $\ltwo{\pbb}=1$.
    I misspoke in class when I said that the L2 norm of a matrix equals its top eigenvalue;
    instead, it equals it's top \emph{singular value}.
    The singular values of a matrix $A$ are defined to be the square root of the eigenvalues of $\trans A A$.
    %I misunderstood
    For real, symmetric matrices, the singular values and the eigenvalues are always the same;
    otherwise, they do not have to be the same (as in the case of the $\pbb$ example above.


    %it is not true that having a maximum eigenvalue of 1 is sufficient for for the steps above to be true.
    %Instead, we need the maximum \emph{singular value} to be 1.
    %The singular values are closely related to eigenvalues,
    %but the maximum singular value of a stochastic matrix can be greater than 1 (as the counter example above shows).
    %For many matrices, the singular values equal the eigenvalues,
    %but it is not true in general that this will be the case for the $\pbb$ matrix.
    %I believe I misspoke in class at some point and stated that the maximum eigenvalue being 1 implies that the above inequality,
    %and so that is why I am offering full credit for the incorrect ``solution'' to this problem,
    %and why I am giving everyone credit for problem 1.4 in the T/F/O questions.

    If you said the answer was TRUE, and it looked to me like your justification was something like this, but you didn't explicitly state anywhere that the max eigenvalue is 1 or that $\ltwo{\pbb}=1$, then you got 0.5 points.
\end{solution}

\newpage
\begin{problem}
%Equation 5.1 shows the power method iteration for solving for $\pr$.
%It is reproduced below
    Recall that Equation (5.1) states that
\begin{equation*}
    \x^{(k)T}
    =
    \alpha \x^{(k-1)T} \p + \big(\alpha \x^{(k-1)T} \mathbf a + (1-\alpha)\big) \mathbf v^T
    .
    \label{eq:xk}
\end{equation*}
    %\begin{enumerate}
        %\item
            In class, we computed the runtime for a single iteration of Equation (5.1) assuming that $\p$ is sparse.
What is the runtime of computing a single iteration if $\p$ is dense?
            %\vspace{4in}
%
        %\item
    \end{problem}
    \begin{solution}
        $O(n^2)$

        If you didn't simplify the solution, you received +0.5
    \end{solution}

    \newpage
    \begin{problem}
        If $\p$ is dense (as in the previous problem), then what is the overall runtime for using the power method to compute an approximate pagerank vector with residual accuracy $\epsilon$?
            %how many iterations are required to achieve residual accuracy
            %\begin{equation*}
               %\ltwo{\x^{(k)} - \x^{(k-1)}} \le \epsilon,
            %\end{equation*}
            %where $\epsilon$ is a ``small'' target accuracy satisfying $0 < \epsilon < 1$.
    %\end{enumerate}
\end{problem}
    \begin{solution}
        $O\bigg(\frac{\log\epsilon}{\log\alpha}n^2\bigg)$

        If you didn't simplify the solution, you received +0.5

        If you used the incorrect value from the previous solution, then you received +0.5
    \end{solution}

\newpage
\begin{problem}
    In practice, no one uses the EAPM to compute pagerank vectors.  Why?
\end{problem}
    \begin{solution}
        The runtime for the EAPM is $O\big((\log \frac{\log\epsilon}{\log\alpha})n^3\big)$ compared to $O\big(\frac{\log\epsilon}{\log\alpha}(\nnz(P)+n)\big)$ for the standard PM.
        The EAPM has better dependencies on $\epsilon$, but much worse dependencies on $n$.
        In real-world problems, $\epsilon$ is always less than about $10^{-6}$, but $n$ can grow extremely large.
        Therefore, we care about the dependence on $n$ much more than the dependence on $\epsilon$.
    \end{solution}

\newpage
\begin{problem}
    Let $A$ be an $n\times n$ dense matrix and $k$ be a positive integer.
    Describe a procedure for computing $A^k$ in time $O(n^3\log k)$.
\end{problem}
    \begin{solution}
        In class, we used the recursive formula
        \begin{equation}
        \label{eq:Qk}
        \Q_k = 
        \begin{cases}
            \pbb & \text{if}~k=0 \\
            \Q_{k-1} \Q_{k-1} & \text{otherwise} \\
        \end{cases}
        \end{equation}
        to compute the $\pbb^k$ expression quickly in the EAPM, and any variation on this earned full credit.

        Technically, the procedure above only works for values of $k$ that are powers of 2, and the question is asking for a procedure that works for any $k$.
        The following recursion computes $A^k$ for all values of $k$:
    \begin{equation}
        \label{eq:Qk}
        A^k = 
        \begin{cases}
            1 & \text{if}~k=0 \\
            A \cdot (A^2)^{(k-1)/2} & \text{if $k$ is odd} \\
            (A^2)^{k/2} & \text{if $k$ is even} \\
            %\Q_{k-1} \Q_{k-1} & \text{otherwise} \\
        \end{cases}
        .
%if n < 0  then return exp_by_squaring(1 / x, -n);
%else if n = 0  then return  1;
%else if n is even  then return exp_by_squaring(x * x,  n / 2);
%else if n is odd  then return x * exp_by_squaring(x * x, (n - 1) / 2);
%
    \end{equation}
    \end{solution}

\newpage
\begin{problem}
    Describe a problem that can be solved using pagerank.
    (This is similar to Problem 1 from the Pagerank III notes, but your example problem should be different than the ones we did in class.)
\end{problem}
\begin{solution}
    Any remotely reasonable solution earned full credit.
\end{solution}
\end{document}




