\documentclass[10pt]{exam}

\usepackage[margin=1in]{geometry}
\usepackage{amsmath}
\usepackage{amssymb}
\usepackage{amsthm}
\usepackage{mathtools}
\usepackage{bm}
\usepackage[normalem]{ulem}
\usepackage{stmaryrd}

\usepackage{color}
\usepackage{colortbl}
\definecolor{deepblue}{rgb}{0,0,0.5}
\definecolor{deepred}{rgb}{0.6,0,0}
\definecolor{deepgreen}{rgb}{0,0.5,0}
\definecolor{gray}{rgb}{0.7,0.7,0.7}

\usepackage{hyperref}
\hypersetup{
  colorlinks   = true, %Colours links instead of ugly boxes
  urlcolor     = black, %Colour for external hyperlinks
  linkcolor    = blue, %Colour of internal links
  citecolor    = blue  %Colour of citations
}

%%%%%%%%%%%%%%%%%%%%%%%%%%%%%%%%%%%%%%%%%%%%%%%%%%%%%%%%%%%%%%%%%%%%%%%%%%%%%%%%

\newcommand*{\hl}[1]{\colorbox{yellow}{#1}}

\newcommand*{\answerLong}[2]{
    \ifprintanswers{\hl{#1}}
\else{#2}
\fi
}

\newcommand*{\answer}[1]{\answerLong{#1}{~}}

\newcommand*{\TrueFalse}[1]{%
\ifprintanswers
    \ifthenelse{\equal{#1}{T}}{%
        %\hl{\textbf{TRUE}}\hspace*{14pt}False
        \hl{\texttt{True}}\hspace*{20pt}\texttt{False}\hspace*{20pt}\texttt{Open}
    }{
        \ifthenelse{\equal{#1}{F}}{
        %True\hspace*{14pt}\hl{\textbf{FALSE}}
        \texttt{True}\hspace*{20pt}\hl{\texttt{False}}\hspace*{20pt}\texttt{Open}
        }
        {
            \texttt{True}\hspace*{20pt}{\texttt{False}}\hspace*{20pt}\hl{\texttt{Open}}
        }
    }
\else
    \texttt{True}\hspace*{20pt}\texttt{False}\hspace*{20pt}\texttt{Open}
\fi
} 
%% The following code is based on an answer by Gonzalo Medina
%% https://tex.stackexchange.com/a/13106/39194
\newlength\TFlengthA
\newlength\TFlengthB
\settowidth\TFlengthA{\hspace*{1.8in}}
\newcommand\TFQuestion[2]{%
    \setlength\TFlengthB{\linewidth}
    \addtolength\TFlengthB{-\TFlengthA}
    \noindent
    \parbox[t]{\TFlengthA}{\TrueFalse{#1}}\parbox[t]{\TFlengthB}{#2}
    \vspace{0.25in}
}

%%%%%%%%%%%%%%%%%%%%%%%%%%%%%%%%%%%%%%%%%%%%%%%%%%%%%%%%%%%%%%%%%%%%%%%%%%%%%%%%

\theoremstyle{definition}
\newtheorem{problem}{Problem}
\newtheorem{theorem}{Theorem}
\newtheorem{defn}{Definition}
\newtheorem{refr}{References}
\newcommand{\E}{\mathbb E}
\newcommand{\R}{\mathbb R}
\DeclareMathOperator{\nnz}{nnz}
\DeclareMathOperator{\sign}{sign}
\DeclareMathOperator{\determinant}{det}
\DeclareMathOperator{\Var}{Var}
\DeclareMathOperator{\rank}{rank}
\DeclareMathOperator*{\argmin}{arg\,min}
\DeclareMathOperator*{\argmax}{arg\,max}

\newcommand{\I}{\mathbf I}
\newcommand{\Q}{\mathbf Q}
\newcommand{\p}{\mathbf P}
\newcommand{\pb}{\bar {\p}}
\newcommand{\pbb}{\bar {\pb}}
\newcommand{\pr}{\bm \pi}

\newcommand{\trans}[1]{{#1}^{T}}
\newcommand{\loss}{\ell}
\newcommand{\w}{\mathbf w}
\newcommand{\x}{\mathbf x}
\newcommand{\y}{\mathbf y}
\newcommand{\lone}[1]{{\lVert {#1} \rVert}_1}
\newcommand{\ltwo}[1]{{\lVert {#1} \rVert}_2}
\newcommand{\lp}[1]{{\lVert {#1} \rVert}_p}
\newcommand{\linf}[1]{{\lVert {#1} \rVert}_\infty}
\newcommand{\lF}[1]{{\lVert {#1} \rVert}_F}

\newcommand{\Ein}{E_{\text{in}}}
\newcommand{\Eout}{E_{\text{out}}}
\newcommand{\Etest}{E_{\text{test}}}
%\newcommand{\I}{\mathbf I}
%\newcommand{\Q}{\mathbf Q}
%\newcommand{\p}{\mathbf P}
%\newcommand{\pb}{\bar {\p}}
%\newcommand{\pbb}{\bar {\pb}}
%\newcommand{\pr}{\bm \pi}
\newcommand{\mH}{m_{\mathcal H}}
\newcommand{\dvc}{{d_{\text{VC}}}}
\newcommand{\HH}[1]{\mathcal H_{\text{#1}}}
\newcommand{\Hbinary}{\HH_{\text{binary}}}
\newcommand{\Haxis}{\HH_{\text{axis}}}
\newcommand{\Hperceptron}{\HH_{\text{perceptron}}}


\newcommand{\ignore}[1]{}

%%%%%%%%%%%%%%%%%%%%%%%%%%%%%%%%%%%%%%%%%%%%%%%%%%%%%%%%%%%%%%%%%%%%%%%%%%%%%%%%

\printanswers

\begin{document}


\begin{center}
{
\Huge
    Chapter 3 Quiz Practice Problems
}
\end{center}

\begin{problem}
    For each statement below,
    circle \texttt{True} if the statement is known to be true,
    \texttt{False} if the statement is known to be false,
    and \texttt{Open} if the statement is not known to be either true or false.

\begin{enumerate}
    \item\TFQuestion{T}{You have trained a perceptron model that has high generalization error.  VC theory predicts that applying the PCA feature embedding with a low output dimension will reduce the generalization error with high probability.}

    \item\TFQuestion{F}{You have trained a perceptron model that has high in sample error.  VC theory predicts that applying the PCA feature embedding with a low output dimension will reduce the in sample error with high probability.}

    \item\TFQuestion{T}{Let $\HH{perceptron}$ be the perceptron hypothesis class and let $\HH{stump}$ be the decision stump hypothesis class.  VC theory predicts that the generalization error of $\HH{stump}$ will be lower than the generalization error of $\HH{perceptron}$ with high probability.}

    \item\TFQuestion{T}{Let $\mathcal H_{\tilde d}$ be the perceptron hypothesis class with the PCA feature map of dimension $\tilde d$. If $k$ is a breakpoint for $\mathcal H_{1}$, then $k$ will also be a breakpoint for $\mathcal H_{2}$.}

    \item\TFQuestion{T}{Let $\mathcal H_{Q}$ be the perceptron hypothesis class with the polynomial feature map of degree $Q$. If $k$ is a breakpoint for $\mathcal H_{1}$, then $k$ will also be a breakpoint for $\mathcal H_{2}$.}

    \item\TFQuestion{F}{VC theory predicts that when using the perceptron hypothesis class, centering your data points results in a better generalization error with high probability.}

    %\item\TFQuestion{T}{Let $\mathcal H_{\tilde d}$ be the perceptron hypothesis class with the PCA feature map of dimension $\tilde d$.  You have trained an empirical risk minimizer $g_5$ on $\mathcal H_{5}$ and discovered it has a high in sample error and a low generalization error.  VC theory predicts that the empirical risk minimizer $g_6$ of the hypothesis class $\mathcal H{6}$ will have a lower}

    \item\TFQuestion{F}{Let $\HH{perceptron}$ be the perceptron hypothesis class and let $\HH{stump}$ be the decision stump hypothesis class.  Let $g_{\text{perceptron}}\in\HH{perceptron}$ and $g_{\text{stump}} \in \HH{stump}$ be the empirical risk minimizers.  VC theory predicts that $\Eout(g_{\text{stump}}) \le \Eout(g_{\text{perceptron}})$ with high probability.}

    \item\TFQuestion{T}{Let $\mathcal H$ be the perceptron hypothesis class and let $\mathcal H_\Phi$ be the perceptron hypothesis class with the polynomial feature map of degree 2.  Let $g\in\mathcal H$ and $g_\Phi \in \mathcal H_\Phi$ be the empirical risk minimizers trained on a very large dataset.  Then we are guaranteed that $\Ein(g) \ge \Ein(g_\Phi)$.}

    \item\TFQuestion{F}{Let $g_{\text{axis2}} \in\mathcal H_{\text{axis2}}$ be the empirical risk minimizer for the $\HH{axis2}$ hypothesis class and $g_{\text{stump}} \in \HH{stump}$ be the empirical risk minimizer for the decision stump hypothesis class.  Then we are guaranteed that $\Ein(g_{\text{axis2}}) \le \Ein(g_{\text{stump}})$.}

    \item\TFQuestion{T}{Let $X$ be a dataset shattered by the perceptron hypothesis class.  Then $X$ is guaranteed to also be shattered by the perceptron hypothesis class with the polynomial feature map of degree 7.}

    \item\TFQuestion{F}{Let $X$ be a dataset shattered by the perceptron hypothesis class.  Then $X$ is guaranteed to also be shattered by the decision stump hypothesis class.}

    \item\TFQuestion{T}{Let $X$ be a dataset shattered by the decision stump hypothesis class.  Then $X$ is guaranteed to also be shattered by the perceptron hypothesis class with polynomial feature map of degree 2.}

    \item\TFQuestion{T}{There exists a dataset of size $N=2$ in $d=2$ dimensions that can be shattered by the perceptron hypothesis class but cannot be shattered by the $\HH{axis2}$ hypothesis class.}

    \item\TFQuestion{F}{There exists a dataset of size $N=2$ in $d=2$ dimensions that can be shattered by the perceptron hypothesis class but cannot be shattered by the decision stump hypothesis class.}

    \item\TFQuestion{F}{Define the hypothesis class of concentric circles with the feature map $\Phi$ as
        $$
        \HH{circles,$\Phi$} = \bigg\{ \x \mapsto \big\llbracket\ltwo{\Phi(\x)} \ge \alpha \big\rrbracket: \alpha \in \mathbb R \bigg\}.
        $$
        You have trained an empirical risk minimizer on this hypothesis class using the polynomial feature map of degree $Q=20$;
        the resulting hypothesis has high generalization error.
        VC theory predicts that reducing the degree of the polynomial to $Q=2$ will also reduce the generalization error with high probability.
        }

    \item\TFQuestion{T}{Define the $\HH{axis2}$ hypothesis class with the feature map $\Phi: d \to \tilde d$ as
        $$
        \HH{axis2,$\Phi$} = \bigg\{ \x \mapsto \sigma\sign(\phi(\x)_i) : \sigma \in\{+1, -1\}, i \in [\tilde d] \bigg\}.
        $$
        You have trained an empirical risk minimizer on this hypothesis class using the polynomial feature map of degree $Q=20$;
        the resulting hypothesis has high generalization error.
        VC theory predicts that reducing the degree of the polynomial to $Q=2$ will also reduce the generalization error with high probability.
        }

    %\item\TFQuestion{F}{The PCA feature map can be used to increase the VC dimension of a hypothesis class.
        %}


\end{enumerate}
\end{problem}
\end{document}




