\documentclass[10pt]{exam}

\usepackage[margin=1in]{geometry}
\usepackage{amsmath}
\usepackage{amssymb}
\usepackage{amsthm}
\usepackage{mathtools}
\usepackage{bm}
\usepackage{stmaryrd}
\usepackage{etoolbox}

\usepackage{color}
\usepackage{colortbl}
\definecolor{deepblue}{rgb}{0,0,0.5}
\definecolor{deepred}{rgb}{0.6,0,0}
\definecolor{deepgreen}{rgb}{0,0.5,0}
\definecolor{gray}{rgb}{0.7,0.7,0.7}

\usepackage{hyperref}
\hypersetup{
  colorlinks   = true, %Colours links instead of ugly boxes
  urlcolor     = black, %Colour for external hyperlinks
  linkcolor    = blue, %Colour of internal links
  citecolor    = blue  %Colour of citations
}

\usepackage{listings}

%%%%%%%%%%%%%%%%%%%%%%%%%%%%%%%%%%%%%%%%%%%%%%%%%%%%%%%%%%%%%%%%%%%%%%%%%%%%%%%%

\theoremstyle{definition}
\newtheorem{problem}{Problem}
\newtheorem{example}{Example}
\newtheorem{defn}{Definition}
\newtheorem{fact}{Fact}
\newtheorem{refr}{References}
\newtheorem{theorem}{Theorem}
\newcommand{\E}{\mathbb E}
\newcommand{\R}{\mathbb R}
\DeclareMathOperator{\nnz}{nnz}
\DeclareMathOperator{\sign}{sign}
\DeclareMathOperator{\determinant}{det}
\DeclareMathOperator{\Var}{Var}
\DeclareMathOperator{\rank}{rank}
\DeclareMathOperator{\prob}{\mathbb P}
\DeclareMathOperator*{\argmin}{arg\,min}
\DeclareMathOperator*{\argmax}{arg\,max}

\newcommand{\Ein}{E_{\text{in}}}
\newcommand{\Eout}{E_{\text{out}}}
\newcommand{\Etest}{E_{\text{test}}}
\newcommand{\Ntest}{N_{\text{test}}}
\newcommand{\I}{\mathbf I}
\newcommand{\Q}{\mathbf Q}
\newcommand{\p}{\mathbf P}
\newcommand{\pb}{\bar {\p}}
\newcommand{\pbb}{\bar {\pb}}
\newcommand{\pr}{\bm \pi}

\newcommand{\trans}[1]{{#1}^{T}}
\newcommand{\loss}{\ell}
\newcommand{\w}{\mathbf w}
\newcommand{\wstar}{{\w}^{*}}
\newcommand{\x}{\mathbf x}
\newcommand{\y}{\mathbf y}
\newcommand{\lone}[1]{{\lVert {#1} \rVert}_1}
\newcommand{\ltwo}[1]{{\lVert {#1} \rVert}_2}
\newcommand{\lp}[1]{{\lVert {#1} \rVert}_p}
\newcommand{\linf}[1]{{\lVert {#1} \rVert}_\infty}
\newcommand{\lF}[1]{{\lVert {#1} \rVert}_F}

\newcommand{\mH}{m_{\mathcal H}}
\newcommand{\dvc}{{d_{\text{VC}}}}
\newcommand{\HH}[1]{\mathcal H_{\text{#1}}}
\newcommand{\Hbinary}{\HH_{\text{binary}}}
\newcommand{\Haxis}{\HH_{\text{axis}}}
\newcommand{\Hperceptron}{\HH_{\text{perceptron}}}


\newcommand{\ignore}[1]{}

\renewcommand{\solutiontitle}{}
%\AfterEndEnvironment{solutionorbox}{
    %\ifprintanswers\vspace{\stretch{1}}
    %\fi}

%%%%%%%%%%%%%%%%%%%%%%%%%%%%%%%%%%%%%%%%%%%%%%%%%%%%%%%%%%%%%%%%%%%%%%%%%%%%%%%%

%\printanswers

\begin{document}


\begin{center}
{
\Huge
Quiz: Chapter 1 definitions
}
\end{center}

\begin{center}
%\includegraphics[height=3in]{scooby}
%~~~~~~~~~~
%\includegraphics[height=3in]{ml}
\end{center}

\begin{defn}
    The \emph{perceptron hypothesis class} is defined to be
    \begin{solutionorbox}[1.7in]
        \begin{equation*}
        \mathcal H = \bigg\{ \x \mapsto \sign(\trans\w \x) : \w \in \R^d \bigg\},
        \end{equation*}
        where
        \begin{equation*}
        \sign(a) =
            \begin{cases}
                +1 & \text{if~} a>0 \\
                -1 & \text{if~} a<0
                .
        \end{cases}
        \end{equation*}
    \end{solutionorbox}
\end{defn}

\begin{defn}
    The \emph{in-sample error} is defined to be
    \begin{solutionorbox}[1.7in]
    \begin{equation*}
        \Ein(h) = \frac1N \sum_{i=1}^N \llbracket h(\x_i) \ne y_i \rrbracket
        .
    \end{equation*}
    \end{solutionorbox}
\end{defn}

\begin{defn}
    The \emph{out-of-sample error} is defined to be
    \begin{solutionorbox}[1.7in]
    \begin{equation*}
        \Eout(h) = \prob\big(h(\x) \ne y\big)
        .
    \end{equation*}
    \end{solutionorbox}
\end{defn}

\begin{defn}
    The \emph{test error} is defined to be
    \begin{solutionorbox}[1.7in]
    \begin{equation*}
    \Etest(h) 
    =
    \
    \frac1{\Ntest} \sum_{i=1}^{\Ntest} \llbracket h(\x_i) \ne y_i \rrbracket
        .
    \end{equation*}
    \end{solutionorbox}
\end{defn}

\begin{defn}
    The \emph{generalization error} of a hypothesis $g$ is defined to be
    \begin{solutionorbox}[1.7in]
    $$
        |\Ein(g) - \Eout(g)|.
    $$
    \end{solutionorbox}
\end{defn}

\begin{defn}
    The \emph{true label function} is defined to be
    \begin{solutionorbox}[1.7in]
    \begin{equation*}
        f = \argmin_{h \in \mathcal H^*} \Eout(h),
    \end{equation*}
    where $\mathcal H^*$ is the union of all hypothesis classes.
    \end{solutionorbox}
\end{defn}

\begin{defn}
    The \emph{true error} (also called the \emph{bayes error}) is defined to be
    \begin{solutionorbox}[1.7in]
    \begin{equation*}
        \Eout(f).
    \end{equation*}
    \end{solutionorbox}
\end{defn}

\end{document}



