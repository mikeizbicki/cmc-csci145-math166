\documentclass[10pt]{article}

\usepackage[margin=1in]{geometry}
\usepackage{amsmath}
\usepackage{amssymb}
\usepackage{amsthm}
\usepackage{mathtools}
\usepackage{bm}

\usepackage{color}
\usepackage{colortbl}
\definecolor{deepblue}{rgb}{0,0,0.5}
\definecolor{deepred}{rgb}{0.6,0,0}
\definecolor{deepgreen}{rgb}{0,0.5,0}
\definecolor{gray}{rgb}{0.7,0.7,0.7}

\usepackage{hyperref}
\hypersetup{
  colorlinks   = true, %Colours links instead of ugly boxes
  urlcolor     = black, %Colour for external hyperlinks
  linkcolor    = blue, %Colour of internal links
  citecolor    = blue  %Colour of citations
}

%%%%%%%%%%%%%%%%%%%%%%%%%%%%%%%%%%%%%%%%%%%%%%%%%%%%%%%%%%%%%%%%%%%%%%%%%%%%%%%%

\theoremstyle{definition}
\newtheorem{problem}{Problem}
\newtheorem{defn}{Definition}
\newcommand{\E}{\mathbb E}
\newcommand{\R}{\mathbb R}
\DeclareMathOperator{\nnz}{nnz}
\DeclareMathOperator{\determinant}{det}
\DeclareMathOperator{\Var}{Var}
\DeclareMathOperator{\rank}{rank}
\DeclareMathOperator*{\argmin}{arg\,min}
\DeclareMathOperator*{\argmax}{arg\,max}

\newcommand{\I}{\mathbf I}
\newcommand{\Q}{\mathbf Q}
\newcommand{\p}{\mathbf P}
\newcommand{\pb}{\bar {\p}}
\newcommand{\pbb}{\bar {\pb}}
\newcommand{\pr}{\bm \pi}

\newcommand{\trans}[1]{{#1}^{T}}
\newcommand{\loss}{\ell}
\newcommand{\w}{\mathbf w}
\newcommand{\x}{\mathbf x}
\newcommand{\y}{\mathbf y}
\newcommand{\lone}[1]{{\lVert {#1} \rVert}_1}
\newcommand{\ltwo}[1]{{\lVert {#1} \rVert}_2}
\newcommand{\lp}[1]{{\lVert {#1} \rVert}_p}
\newcommand{\linf}[1]{{\lVert {#1} \rVert}_\infty}
\newcommand{\lF}[1]{{\lVert {#1} \rVert}_F}

%%%%%%%%%%%%%%%%%%%%%%%%%%%%%%%%%%%%%%%%%%%%%%%%%%%%%%%%%%%%%%%%%%%%%%%%%%%%%%%%

\begin{document}


\begin{center}
\Huge
Notes: Pagerank
\end{center}

\begin{center}
\includegraphics[width=\textwidth]{dilbert}
\end{center}

\section{Pre-lecture Work}

\begin{problem}
    Matt Cutts was formerly the head of Google's web spam team,
    and now runs the United States Digital Service (a recently created branch of the US government).
    Watch his video on ``How Google Search Works.''
    \begin{quote}
    \url{https://www.youtube.com/watch?v=KyCYyoGusqs}
    \end{quote}
\end{problem}

\begin{problem}
    Our primary text for this week is \emph{Deeper Inside Pagerank} by Langville and Meyers.
    Read sections 1, 2, 3, 5.1, 6.1, 6.2, 9, 10.
    \begin{quote}
    \url{https://galton.uchicago.edu/~lekheng/meetings/mathofranking/ref/langville.pdf}
    \end{quote}
    Based on the reading, define the following terms:
    \begin{enumerate}
        \item markov chain
            \vspace{1in}
        \item stationary vector
            \vspace{1in}
        \item stochastic matrix
            \vspace{1in}
        \item irreducible matrix
            \vspace{1in}
        \item primitive matrix
            \vspace{1in}
        \item aperiodic markov chain
            \vspace{1in}
        \item $\p$
            \vspace{1in}
        \item $\pb$
            \vspace{1in}
        \item $\pbb$
            \vspace{1in}
        \item $\pr$
            \vspace{1in}
        \item dangling nodes
            \vspace{1in}
        \item personalization vector
            \vspace{1in}
    \end{enumerate}
\end{problem}


%\begin{problem}
    %(optional)
    %Pagerank has many applications outside of web search,
    %and this problem gives you some additional reading that explores these problems at a high level.
%
    %Read the case study on Twitter's WTF system.
    %And yes, that's it's real name.
    %\begin{quote}
    %\url{https://dl.acm.org/doi/10.1145/2488388.2488433}
    %\end{quote}
    %For more applications of pagerank outside of web search, read section 4 of ``Pagerank Beyond the Web.''
    %\begin{quote}
    %\url{https://arxiv.org/abs/1407.5107}
    %\end{quote}
%\end{problem}

\begin{problem}
    Answer the following questions.
    \begin{enumerate}
        \item 
            Is the matrix $\p$ stochastic? irriducible? primitive?
            \vspace{2in}
        \item 
            Is the matrix $\pb$ stochastic? irriducible? primitive?
            \vspace{2in}
        \item 
            Is the matrix $\pbb$ stochastic? irriducible? primitive?
            \vspace{2in}
    \end{enumerate}
\end{problem}

\newpage
\begin{problem}
    Either prove or give a counterexample to the following claims.
    \begin{enumerate}
        \item
            $\rank(\pbb) = 1$.
            \vspace{3in}
        \item
            $\rank(\pbb) = n$.
            \vspace{3in}

        \item
            Let $\x$ be an eigenvector of $\p$ with eigenvalue $\lambda$.
            Then $\frac 1 2 \x$ is also an eigenvector of $\p$ with eigenvalue $\frac 1 2 \lambda$.
            \vspace{3in}

        \item
            The smallest eigenvalue of $\pbb$ is exactly 0.
            \vspace{3in}
        \item
            The largest eigenvalue of $\pbb$ is exactly 1.
            \vspace{3in}
        \item
            The largest eigenvalue of $\p$ is exactly 1.
            \vspace{3in}

        \item
            The largest eigenvector of $\p$ is simple.
            Recall that a simple eigenvalue has multiplicity 1.
            That is, there is exactly 1 eigenvector with the same eigenvalue.
            \vspace{3in}

        \item
            The largest eigenvector of $\pb$ is simple.
            \vspace{3in}

        \item
            The largest eigenvector of $\pbb$ is simple.
            \vspace{3in}

        \item
            The eigenvectors of $\pbb$ are orthogonal.
            \vspace{3in}

        \item
            The eigenvectors of $\pbb\trans\pbb$ are orthogonal.
            \vspace{3in}

        \item
            The eigenvectors of $\trans\pbb\pbb$ are orthogonal.
            \vspace{3in}
    \end{enumerate}
\end{problem}

%%%%%%%%%%%%%%%%%%%%%%%%%%%%%%%%%%%%%%%%%%%%%%%%%%%%%%%%%%%%%%%%%%%%%%%%%%%%%%%%

\newpage
\section{Lecture}

%%%%%%%%%%%%%%%%%%%%%%%%%%%%%%%%%%%%%%%%

\begin{problem}
    The purpose of the pagerank vector $\pr$ is to provide a ranking of of how important a node is.
    There are many alternative ways to provide such a ranking.
    One simple alternative is to rank nodes by their in-degree.
    For ``typical'' graphs, the in-degree ranking and the pagerank ranking will be similar,
    but there are graphs for which the two rankings can be arbitrarily different from each other.

    Draw a graph such that the top ranked node according to pagerank is the bottom ranked node according to in-degree.
\end{problem}

%%%%%%%%%%%%%%%%%%%%%%%%%%%%%%%%%%%%%%%%

\clearpage
\begin{problem}
    The beginning of Section 5 shows the following equivalent definitions for the pagerank vector $\pr$:
    \begin{equation}
        \trans \pr \pbb = \trans \pr
        \qquad
        \text{and}
        \qquad
        \trans \pr (\I - \pbb) = \trans{\bm 0}
        .
    \end{equation}
    It should be obvious why these definitions of $\pr$ are equivalent. 
    Less obvious (and not shown in the paper) is that
    the following definition is also equivalent.
    Prove this equivalence.

    \begin{equation}
        \pr = \argmax_{\w \in \R^d, \ltwo{\w} \le 1} \ltwo{\trans \w \pbb}
    \end{equation}
\end{problem}

%%%%%%%%%%%%%%%%%%%%%%%%%%%%%%%%%%%%%%%%

\newpage

\begin{problem}
    In this question you will calculate the runtime of the power method for computing pagerank.
    Assume that $P$ is a sparse matrix and that $\pr$ is dense.

    \begin{enumerate}
        \item
            Equation 5.1 shows the power method iteration for solving for $\pr$.
            It is reproduced below
            \begin{equation}
                \x^{(k)T}
                =
                \alpha \x^{(k-1)T} \p + (\alpha \x^{(k-1)T} \mathbf a + (1-\alpha)) \mathbf v^T
                .
                \label{eq:xk}
            \end{equation}
            What is the runtime of computing $x^{(k)}$ from $x^{(k-1)}$? % using Equation \eqref{eq:xk} if $\pr$ is stored as a sparse matrix in COO format?
            \vspace{4in}

        \item 
            \label{item:2}
            Given only $\x^{(0)}$, what is the runtime of computing $\x^{(K)}$ by iterating Equation $\eqref{eq:xk}$ $K$ times?
            \vspace{4in}

        \item
            \label{item:3}
            When computing pagerank,
            we typically do not know the final number of iterations $K$ in advance.
            Instead, we continue our computation until the following condition is met:
            \begin{equation}
                \ltwo { \x^{(k)} - \x^{(k-1)} } \le \epsilon,
            \end{equation}
            where $\epsilon$ is a ``small'' number that controls how accurate we want our solution to be.
            Compute a formula for $K$ as a function of $\epsilon$.
            \vspace{4in}

        \item
            Substitute your answer for part \ref{item:3} into your answer for part \ref{item:2} to get a formula for the overall runtime in terms of the final desired accuracy $\epsilon$.
            \vspace{4in}

        \item
            Now assume that $\pbb$ is stored as a dense matrix.
            Repeat the calculations for the runtime of the power method in terms of the desired accuracy $\epsilon$.
            \vspace{4in}

        \item
            We say that an algorithm for computing the pagerank \emph{converges} if in the limit as the number of iterations goes to infinity, the algorithm returns the correct pagerank vector.
            Have we shown that the power method converges?
            Or are there conditions in which is will \emph{diverge} (i.e.\ not converge)?
            \vspace{2in}

        \item
            Why does it never make sense to store $\pr$ as a sparse vector?
            \vspace{2in}

        \item
            \label{prob:comp:k}
            Prove the following statement or provide a counterexample.
            \begin{equation}
                \ltwo{\x^{(k)}} < \ltwo{\x^{(k-1)}}
                .
            \end{equation}
            \vspace{4in}
        \item
            Based on the results of part \ref{prob:comp:k} above,
            how should we adjust our implementation of the power method?

    \end{enumerate}
\end{problem}

%%%%%%%%%%%%%%%%%%%%%%%%%%%%%%%%%%%%%%%%

\newpage
\begin{problem}
    There are many alternative algorithms for computing pagerank vectors.
    In this problem, we will investigate an algorithm that I call the \emph{exponentially accelerated power method},
    although it does not have a commonly accepted name.
    This is a divide and conquer algorithm that can achieve the same accuracy $\epsilon$ as the power method with only a logarithmic number of iterations.

    The estimated pagerank vector is given by
    \begin{equation}
        \y^{(K)} = \x^{(0)} \Q_K
        ,
    \end{equation}
    where
    \begin{equation}
        \Q_k = 
        \begin{cases}
            \pbb & \text{if}~k=0 \\
            \Q_{k-1} \Q_{k-1} & \text{otherwise} \\
        \end{cases}
        .
    \end{equation}
    In the standard power method, the matrix $\pbb$ is not stored explicitly,
    but is calculated from the $\p$ matrix.
    In this problem, you can assume for simplicity that the $\pbb$ matrix is stored explicitly as a dense matrix,
    and that $\Q_k$ is also stored as a dense matrix.

    \begin{enumerate}
        \item
            Show that $\y^{(K)} = \x^{(2^K)}$.
            This equivalence is why the algorithm is ``exponentially accelerated.''
            \vspace{3in}

        \item
            What is the runtime of calculating $\Q_k$ given $\Q_{k-1}$? 
            \vspace{4in}

        \item 
            What is the runtime of computing $\y^{(K)}$ in terms of $K$?
            \vspace{4.5in}

        \item
            What is the runtime of computing $\y^{(K)}$ in terms of $\epsilon$.
            \vspace{4.5in}

        \item
            Under what conditions is the exponentially accelerated power method faster than the standard power method?
            \vspace{3in}

        \item
            Under what conditions is it slower?
            \vspace{3in}

        \item
            What bad thing would happen if $\p$ was stored as a sparse matrix and $\pbb$ was calculated from $\p$ as in the standard power method?
            \vspace{4in}

    \end{enumerate}
\end{problem}

\end{document}


